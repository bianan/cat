\usepackage{jmlr2e_cus}
\usepackage[usenames]{xcolor}
%\usepackage[usenames]{color}
\usepackage[bookmarksnumbered=true]{hyperref}
\usepackage{url}
%\definecolor{cite_color}{RGB}{0, 51, 153}
\definecolor{cite_color}{RGB}{0, 0, 255}
%\definecolor{link_color}{RGB}{0,102,102}%  green-style
\definecolor{link_color}{RGB}{153, 0,0}  %  red
\definecolor{url_color}{RGB}{153, 102,  0}
\definecolor{emp_color}{RGB}{0,0,255}
% {51, 102, 51}
\hypersetup{
 colorlinks,
 citecolor=cite_color,
 linkcolor=link_color,
 urlcolor=url_color}
 % graph
\graphicspath{{./figs/}}

 
 
\usepackage{epsfig}
\usepackage{amsmath}
\usepackage{amssymb}
\usepackage{mathrsfs}
\usepackage{graphicx}
\usepackage{subfig}
\usepackage{multirow}
\usepackage{booktabs}
%\usepackage{mathrsfs} 
\usepackage{wrapfig}
\usepackage{enumerate}
\usepackage{bm}
%\usepackage{bbm}
% to use \bm in section title 
\pdfstringdefDisableCommands{%
	\renewcommand*{\bm}[1]{#1}%
	% any other necessary redefinitions 
}
\usepackage{tabularx}
% citation 
\usepackage{natbib}
\bibpunct{(}{)}{;}{a}{,}{,}
%1 the opening bracket symbol, default = (
%2 the closing bracket symbol, default = )
%3 the punctuation between multiple citations, default = ;
%4 the letter `n’ for numerical style, or `s’ for numerical superscript style, any other letter for author-year, default = author-year;
%5 the punctuation that comes between the author names and the year
%6 the punctuation that comes between years or numbers when common author lists are suppressed (default = ,);
\bibliographystyle{icml2017}
%  abbrvnat  plainnat icml2017

\usepackage{mathtools}



% For algorithms, use algorithm2e
\usepackage[vlined, ruled, linesnumbered]{algorithm2e}
%set comment style
\newcommand\mycommfont[1]{\footnotesize\textit{
\hspace{-.3em}#1}}
\SetCommentSty{mycommfont}
\SetKwInOut{Input}{input}
\SetKwInOut{Output}{output}

 \usepackage{cleveref} % should be put after other packages 
 %  using always captial letter
 \crefname{section}{Section}{Sections}
 \crefname{theorem}{Theorem}{Theorems}
 \crefname{lemma}{Lemma}{Lemmas}
 \crefname{equation}{Equation}{Equations}
 \crefname{proposition}{Proposition}{Propositions}
 \crefname{claim}{Claim}{Claims}
\crefname{appendix}{Appendix}{Appendices}
   \crefname{algorithm}{Algorithm}{Algorithms}
 \crefname{figure}{Figure}{Figures}
 \crefname{table}{Table}{Tables}
 \crefname{remark}{Remark}{Remarks}
 \crefname{definition}{Definition}{Definitions}
 \crefname{equatinon}{Equation}{Equations}
 \crefname{corollary}{Corollary}{Corollaries}
 %  \crefname{equation}{Equation}{Equations}
 %  \crefname{proposition}{proposition}{propositions}
 %  \crefname{claim}{claim}{claims}
 % usage: 
 %In the first argument of \crefname{}{}{} comes the type of reference (equation, figure, table, section, etc.). The second argument contains the word that is printed if only one reference is made and the third argument contains the plural form for multiple references.


\allowdisplaybreaks
\usepackage{thm-restate}

\def\labelitemi{-}
% set margin of list 
\usepackage{enumitem}
\setlist[itemize]{leftmargin=9mm}


\newcommand{\todo}[1]{\textcolor{red}{@todo:}#1}
\newcommand{\appendixtitle}[1]{
	\begin{center}
		\LARGE \bf #1
	\end{center}
}


\def\C{{\cal C}}
\def\S{{\cal S}}
\def\E{{\mathbb E}}
\def\X{{\cal X}}
\def\Y{{\cal Y}}
\def\H{{\mathbb  H}}
\def\Z{{\mathbb Z}}
\def\L{{\cal L}}
\def\N{{\cal N}}

\def\cone{{\cal K}}

%\def\A{{\mathscr A}}
\def\M{{\cal M}}
\def \I {{\cal{I}}}
\def\T {\mathbb{T}}
\def\opt{\ensuremath{\Omega^*}}
\def\optcont{\ensuremath{\x^*}}
\newcommand{\opti}[1]{x^*_{#1}} % ith entry of opt, used 
% in double greedy proof



% bold small letters
\def \c{\mathbf{c}}
\def \v{\mathbf{v}}
\def \w{\mathbf{w}}
\def \r{\mathbf{r}}
\def \a{\mathbf{a}}
\def \b{\mathbf{b}}
\def \d{\mathbf{d}}
\def \x{\mathbf{x}}
\def \y{\mathbf{y}}
\def \s{\mathbf{s}}
\def \e{\mathbf{e}}
\def \u{\mathbf{u}}
\def \bmu{\mathbf{u}}
\def \t{\mathbf{t}}
\def \z{\mathbf{z}}
\def \h{\mathbf{h}}
\def \bu{\mathbf{u}}


\def \BA{\mathbf{A}}
\def \BB{\mathbf{B}}
\def \BX{\mathbf{X}}
\def \BY{\mathbf{Y}}
\def \BI{\mathbf{I}}
\def \BC{\mathbf{C}}
\def \BD{\mathbf{D}}
\def \BE{\mathbf{E}}
\def \BH{\mathbf{H}}
\def \BL{\mathbf{L}}
\def \BM{\mathbf{M}}
\def \BG{\mathbf{G}}
\def \BP{\mathbf{P}}
\def \BQ{\mathbf{Q}}
\def \BS{\mathbf{S}}
\def \BU{\mathbf{U}}
\def \BV{\mathbf{V}}
\def \BW{\mathbf{W}}
\def \BZ{\mathbf{Z}}
% compatibable with aistats
\def \bmA{\mathbf{A}}
\def \bmH{\mathbf{H}}
\def \bmL{\mathbf{L}}
\def \bmX{\mathbf{X}}
\def \bmY{\mathbf{Y}}
\def \bmI{\mathbf{I}}
\def \bmW{\mathbf{W}}
\def \bmZ{\mathbf{Z}}


\def \BLambda{\mathbf{\Lambda}}  % B or b, depend on the first letter of the symbol
\def \BSigma{\mathbf{\Sigma}}
\def \bsigma{\bm{\sigma}}
\def \btheta{\bm{\theta}}
\def \blambda{\bm{\lambda}}
\def \balpha{\bm\alpha}
\def \bphi{\bm{\phi}}


\def \bmsigma{\bm{\sigma}}
\def \bmtheta{\bm{\theta}}
\def \bmlambda{\bm{\lambda}}
\def \bmalpha{\bm\alpha}
\def \bmphi{\bm{\phi}}
\def \bmchi{\bm{\chi}}


\def \P{{\cal{P}}}
\def \Q{{\cal{Q}}}
\def \W{{\cal{W}}}
\def \B{{\cal{B}}}
\def \R{{\mathbb{R}}}
%  the transpose
\def \trans{\top}
% trace
\newcommand{\tr}[1]{\text{tr}(#1)}
\newcommand{\pare}[1]{{#1}}  % parethese sth, with extra {} added, used in a
% math mode


\def \D{{\cal{D}}}
\def \V{{\cal{V}}}
\def \E{{\mathbb{E}}} %  expection

%\newcommand{\MAXCUT}{{\textsc{MaxCut}}}
%\newcommand{\LOVASZ}{Lov{\'a}sz }

\newcommand{\argmin}{{\arg\min}}
\newcommand{\diag}{{\text{diag}}}
\newcommand{\dom}[1]{{\texttt{dom}(#1)}}
\newcommand{\sign}[1]{{\text{sign}(#1)}}
\newcommand{\dep}{{\text{Dep}}} % dependency set of a set function
\newcommand{\var}{{\text{Var}}} %  variance
\newcommand{\argmax}{{\arg\max}}
\newcommand{\algname}[1]{{\textsc{#1}}}
\newcommand{\note}[1]{\textcolor{blue}{//}#1}
%\newcommand{\spt}[1]{{\texttt{supp}^+}(#1)}
\newcommand{\spt}[1]{{\texttt{supp}}(#1)}
\newcommand{\dtp}[2]{\langle #1, #2\rangle}% dot product
\newcommand{\de}[1]{\text{det}\left(#1\right)}% determinant of a square matrix
\newcommand{\fracpartial}[2]{\frac{\partial #1}{\partial #2}}
\newcommand{\fracppartial}[2]{\frac{\partial^2 #1}{\partial #2}}


\newcommand{\lleq}{\preceq}
\newcommand{\ggeq}{\succeq}
\renewcommand{\ll}{\prec}
\renewcommand{\gg}{\succ}



% new theorem
%
\newtheorem{assumption}{Assumption}
\newtheorem{theorem}{Theorem}
\newtheorem{lemma}{Lemma}
\newtheorem{proposition}{Proposition}
\newtheorem{corollary}{Corollary}
\newtheorem{definition}{Definition}
\newtheorem{remark}{Remark}
\newtheorem{claim}{Claim}
\newtheorem{observation}{Observation}


% floor and ceil 
\DeclarePairedDelimiter\ceil{\lceil}{\rceil}
\DeclarePairedDelimiter\floor{\lfloor}{\rfloor}

% new
\newcommand{\parti}{\text{Z}} % partition function
\newcommand{\zero}{\mathbf{0}} % vector zero 
\newcommand{\one}{\mathbf{1}} % vector zero 
\newcommand{\ele}{v} % element of a ground set 
\newcommand{\multi}{f_{\text{mul}}} % multilinear extension
\newcommand{\prob}{\text{P}} % probability  
\newcommand{\mixingmat}{\BW}
\def\dist{{\cal D}} % distribution 